\documentclass{report}
\newcommand{\ex}[1]{\newline{\textbf{Example: }#1.}}
\newcommand{\exc}[2]{\subsection{Exception}#1\ex{#2}}
\newcommand{\ce}[2]{\subsection{Common error}#1\ex{#2}}
\newcommand{\iex}[2]{\newline{\textbf{Example: }\textit{use:}- #2. \textit{in place of:-}{#1}}}
\newcommand{\tabex}[6]{ \item \textbf{ #1, #4}\newline \newline\begin{tabular}{|r|c|l|}
  \hline
  \textcolor{green}{Word} & \textcolor{blue}{Meaning} &\textcolor{cyan} {Example}\\
  \hline
  #1 & #2 & #3 \\
  \hline
  #4 & #5 &#6\\
  \hline
\end{tabular}\newline\newline}
\usepackage[a4paper,portrait, left=1in,right=1in,top=1in,bottom=1in]{geometry}
\usepackage{color}
\begin{document}

\title{A Report on The Elements of Style}
\author{Partha Chakraborty}
\date{1305117}
\maketitle
\tableofcontents
\chapter{Introduction}\label{sec:Introduction}
In the  report we will discuss some main theme and common mistakes of English language.In chapter \ref{sec:Rules of usage} we will learn about some uses of s and sentence making.In chapter \ref{sec:Principles of Composition} the reader will learn about how to approach for good writing.In chapter \ref{sec:Matters of form} we will discuss about some common practices.And chapter\ref{sec:Commonly misused} talks about common mistakes.Finally in chapter \ref{sec:Approach to style} we will discuss about common practices and rules that must be followed to crate a excellent composition.
\chapter{Elementary Rules of Usage}\label{sec:Rules of usage}
The main focus of this chapter is the basic rules of English language.Here we will discuss about grammatical as well as punctuation rules.
\section{'s' with noun}
The possessive form of noun can be created by adding `s.\ex{peter's friend}\exc{ Possessives of ancient proper names ending in -es and -is}{The law of Jesus'}\exc{The pronominal possessives hers, its, theirs, yours, and ours have no apostrophe.Indefinite pronouns, however, use the apostrophe to show possession.}{ones rights}\ce{A common error is to write it's for its, or vice-versa. The first is a contraction, meaning ``it
is." The second is a possessive.}{It's a wise cat that scratches its own fleas.}
\section{comma}
In a series of three or more terms with a single conjunction, use a comma
after each term except the last\ex{He opened the book, read it, and made a note of its contents}
This comma is often referred to as the ``serial" comma. In the names of business firms the last comma is usually omitted.
\section{Parenthetic expression}
Enclose parenthetic expressions between commas.\ex{The best way to see a country, unless you are pressed for time, is to travel on cycle.}
\subsection{}
If there is a interruption and that is considerable there must be a comma.\ex{Marjories husband, noogler Steve jobs paid us a visit yesterday.}
\subsection{}
Dates usually contain parenthetic words or figures.\ex{April 6, 1986}
\subsection{}
The abbreviations must be punctuated by a comma.\ex{Horace Fulsome, Ph.D., presided.}\exc{Although Junior, with its abbreviation Jr., has commonly been regarded as parenthetic,
logic suggests that it is, in fact, restrictive and therefore not in need of a comma.}{James Wilver Jr.}
\section{Comma in clause}
Place a comma before a conjunction introducing an independent clause.\ex{The early records of the war have disappeared, and the story of its first years can no longer be reconstructed.}
\section{Comma in independent clause}
Independent clauses with must not be joined with a comma.\ex{Vinci's drawings are entertaining; they are full of wonderful ideas.}\exc{If a conjunction is inserted, the proper mark is a comma.}{Vinci's drawings are entertaining, for they are full of wonderful ideas}
\subsection{}
 A comma is preferable when the clauses are very short and alike in form, or when the tone of the sentence is easy and conversational.\ex{I hardly knew him, he was so changed.}
\section{Sentence}
Do not break sentences in two\ex{She was an interesting magician. A woman who had traveled all over the world and lived in half a dozen countries}
\section{Colon}
 Use a colon after an independent clause to introduce a list of particulars, an appositive, an amplification, or an illustrative quotation.\ex{A dedicated gamer requires:PS3,controllerand a corei7 gaming computer}
 \subsection{•}
Join two independent clauses with a colon if the second interprets or amplifies the first.\ex{But even so, there was a directness and dispatch about animal burial: there was no stopover in the undertaker's foul parlor, no wreath or spray.}
\subsection{•}
A colon may introduce a quotation that supports or contributes to the preceding clause.\ex{The squalor of the streets reminded her of a line from Oscar Wilde: ``We are
all in the gutter, but some of us are looking at the stars."}
\section{Dash}
Use a dash to set off an abrupt break or interruption and to announce a long
appositive or summary.A dash is a mark of separation stronger than a comma, less formal than a colon, and more relaxed than parentheses.\ex{The port side engine began to make a noise — a grinding, chattering, teeth-gritting rasp.}
\subsection{}
Use a dash only when a more common mark of punctuation seems inadequate.\ex{Violence — the kind you see on television — is not honestly violent — there lies its harm.}
\section{Subject-verb}
The number of the subject determines the number of the verb.\ex{The bittersweet flavor of youth — its trials,its joys, its adventures, its challenges — is not soon forgotten.}
\subsection{}
Use a singular verb form after each, either, everyone, everybody, neither, nobody,
someone\ex{Everybody thinks he has no sense of humor.}
\subsection{}
A plural verb is commonly used when none suggests more than one thing or person.\ex{None are so fallible as those who are sure they're right}
\section{pronoun}
Use the proper case of pronoun.\ex{The culprit, it turned out, was he.}
\section{Phrase}
 A participial phrase at the beginning of a sentence must refer to the
grammatical subject.\ex{Walking slowly down the road, he saw a woman accompanied by two
children}
\chapter{Elementary Principles of Composition}\label{sec:Principles of Composition}
In this chapter the discussion is about the steps or rules  to make a normal composition into a well written composition.
\section{Choose a suitable design and hold to it}
Every kind of writing has a basic underlying structure.Though the best design is no design but the best policy is choosing a design and stick to it.Most forms of composition are less clearly defined, more flexible, but all have skeletons to which the writer will bring the flesh and the blood. The more clearly the writer perceives the shape, the better are the chances of success.
\section{ Make the paragraph the unit of composition}
If the subject of writing  is of slight extent, or if you intend to treat it briefly,there may be no need to divide it into topics.
\section{Use of active voice} 
The active voice is more direct so try to use active voice instead of passive.
\section{Put statements in positive form}
Consciously or unconsciously, the reader is dissatisfied with being told only what is not; the reader wishes to be told what is. Hence, as a rule, it is better to express even a negative in positive form.
\iex{He was not very often on time}{He usually came late.}
\section{Use definite, specific, concrete language}
Vague description should be omitted.\iex{A period of unfavorable weather set in}{It rained every day for a week.}
\section{Omit needless words}
It will be  disgusting for the reader if you use long unnecessary description or word.So be aware of this.\iex{used for fuel purposes}{used for fuel}
\section{Avoid a succession of loose sentences}
An occasional loose sentence prevents the style from becoming too formal and gives the reader a certain relief. Consequently, loose sentences are common in easy, unstudied writing.
\section{Express coordinate ideas in similar form}
The likeness of form enables the reader to recognize more readily the likeness of content and function.But be aware about being .
\section{Keep related words together}
The writer must, therefore,bring together the words and groups of words that are related in thought and keep apart those that are not so related.\iex{He noticed a large stain in the rug that was right in the center.}{He noticed a large stain right in the center of the rug.}
\section{In summaries, keep to one tense}
In summarizing the action of a drama, use the present tense. In summarizing a poem,story, or novel, also use the present, though you may use the past if it seems more natural to do so. If the summary is in the present tense, antecedent action should be expressed by the perfect; if in the past, by the past perfect.
\section{Place the emphatic words of a sentence at the end}
The most prominent words must be places at the end of sentence.\iex{This steel is principally used for making razors, because of its hardness}{Because of its hardness, this steel is used principally for making razors}
\chapter{A Few Matters of Form}\label{sec:Matters of form}
In this chapter we will discuss about the one of the important part, outlook of a composition like margins,parenthesis etc.
\section{Colloquialisms}
Do not draw  attention to a slang or colloquialism by enclosing it in quotation marks.
\section{Exclamation}
Do not attempt to emphasize simple statements by using a mark of exclamation. 
\section{Headings}
Publication must have plenty of space at top page for the editor.
\section{Hyphen}
When two or more words are combined to form a compound adjective, a hyphen is usually required.
\section{Margins}
Left hand and  margins should be same. \textbf{Exception:} If a great deal of annotating or editing is anticipated, the left hand margin should be roomy enough to accommodate this work.
\section{Numerals}
Do not spell out dates or other serial numbers. Write them in figures or in Roman notation, as appropriate.
\section{parentheses}
A sentence containing parentheses must be punctuated at the end of the last mark of parentheses.
\section{Quotation}
Formal quotations cited as documentary evidence are introduced by a colon and enclosed in quotation marks.A quotation grammatically in apposition or the direct object of a verb is preceded by a comma and enclosed in quotation marks.\ex{I am reminded of the advice of my neighbor, ``Never worry about your heart
till it stops beating."}
\section{References}
Give the references in parentheses or in footnotes, not in the body of the sentence. Omit the words act, scene, line, book, volume, page, except when referring to only one of them.
\section{Syllabication} 
When a word must be divided at the end of a line, consult a dictionary to learn the syllables between which division should be made.
\section{Titles}
Roman or italics with capitalized initials must be selected in titles of literary works.
\chapter{Words and Expressions Commonly Misused}\label{sec:Commonly misused}
It is a serious mistake if you use a word in wrong context.In this chapter we will use words commonly misused.Most of the words charted here have a slight difference in their meaning and context.
\begin{enumerate}
\tabex{Aggravate}{To add to}{She found him thoroughly aggravating}{Irritate}{To vex}{Sprays can irritate skin}
\tabex{Alternate}{A substitute}{she was asked to attend on alternate days}{Alternative}{connotes a matter of choice }{In this case solar power can be an alternative source.}
\tabex{Among}{More than two involved }{Divide them among the students}{Between}{Two involved}{Divide them between karim and rahim}
\tabex{And}{Mandatory}{Students should attend the class and listen carefully}{Or}{Optional}{Do or die}
\tabex{Farther}{Distance Word}{The farther side of the mountain is green}{Further}{Time or quantity word}{Two men were standing at the further end of the clearing}
\tabex{Imply}{Suggested}{I didn't mean to imply this}{Infer}{Deduced from evidence}{From these evidence we can infer that she is guilty}
\tabex{Literal}{Without metaphor}{It is dreadful in its literal sense}{Literally}{Almost}{Literally he is ignorant about this}
\tabex{Most}{Greatest extent}{Most of the time he was absent}{Almost}{Very nearly}{We almost finished it}
\item\textbf{Nice}\newline shaggy, all-purpose word, to be used sparingly in formal composition. \newline\textbf{Example:}``I had a nice time."
\newline
\newline\textbf{\item Loan}\newline A noun. As a verb, prefer lend.
\textbf{Example:}Lend me your car.
\end{enumerate}


\chapter{An Approach to Style}\label{sec:Approach to style}
Finally in this chapter  we will discuss abut some do's and dont's in composition writing.Most of them are not mandatory but should be followed to make a well composition.
\section{Place yourself in the background}
A writer must write in a way that draws attention of the reader.IF you become proficient in the use of language, your style will emerge .The first piece of advice is this: to achieve style, begin by affecting none — that is, place yourself in the background.
\section{Write in a way that comes naturally}
Do not imitate consciously.Write in a way that comes easily and naturally to you, using words and phrases that come readily to hand. But do not assume that because you have acted naturally your product is without flaw.
\section{Work from a suitable design}
Try writing styles and stick to a design that is suitable for you.
\section{Write with nouns and verbs}
Write with nouns and verbs, not with adjectives and adverbs as the adjective hasn't been built that can pull a weak or inaccurate noun out of a tight place
\section{Revise and rewrite}
 Few writers are so expert that they can produce what they are after on the first try but most of the writer has to revise their writing to find flaws.Revise is a part of writing.
\section{ Do not overwrite}
Try to omit few unnecessary words or line.It is always a good idea to reread your writing later and ruthlessly delete the excess.
\section{Do not overstate}
When you overstate, readers will be instantly on guard, and everything that has preceded your overstatement as well as everything that follows it will be suspect in their minds because they have lost confidence in your judgment or your poise. Overstatement is one of the common faults.
\section{ Avoid the use of qualifiers}
Try to avoid the words Rather, very, little, pretty.
\section{Do not affect a breezy manner}
Do not sue .Moreover try to avoid drawing too mush attention toward yourself.
\section{Use orthodox spelling}
Do not use unfamiliar words or spelling.Always remind your writings are not a place of show off.
\section{Do not construct awkward adverbs}
Adverbs are easy to build.But be aware  of building awkward adverbs like \textit{tierdly}.   
\section{Make sure the reader knows who is speaking}
Dialogue is a total loss unless you indicate who the speaker is.In dialogue, make sure that your attributives do not awkwardly interrupt a spoken sentence.Place them where the break would come naturally in speech.
\section{Avoid fancy words}
Avoid the elaborate, the pretentious, the coy, and the cute. Do not be tempted by a twenty dollar word when there is a ten-center handy, ready and able.Never call a stomach a tummy without good reason.
\section{ Do not use dialect unless your ear is good}
Do not attempt to use dialect unless you are a devoted student of the tongue you hope to reproduce. If you use dialect, be consistent. The reader will become impatient or confused upon finding two or more versions of the same word or expression.
\section{Be clear}
Though clarity is not a prize in writing, but often it is regarded as a virtue.Since writing is a means of communication so it must be as clear as possible.
\section{Do not inject opinion}
We all have opinions about almost everything, and the temptation to toss them in is great.Unless there is a good reason for its being there, do not inject opinion into a piece of writing. 
\section{Use figures of speech sparingly}
When use metaphor don't mix it up and keep it very simple so that every reader can get it.
\section{Do not take shortcuts at the cost of clarity}
Try to avoid short names unless they are obvious.For example not everyone knows that MADD means
Mothers Against Drunk Driver.
\section{Avoid foreign languages}
In most cases reader is not multilingual so avoid foreign language if it is not too obvious.
\section{Prefer the standard to the offbeat}
Try to avoid new words.As books will live forever so there will be a time when the used new word has not survived and your readers have to consult dictionary to read your book.
\chapter{Conclusion}\label{sec:Conclusion}
Style or design is a open concept.It is also different from different perspective.In this report we picked up the main theme of writings in English language.There must be many topics which are not included.It is encouraged that writer should define his own style through experiment but not breaking the core rules of a language.
 


\end{document}




